\documentclass[11pt,a4paper]{article}

\usepackage{amsmath,amssymb}
\usepackage{graphicx}
\usepackage{booktabs}
\usepackage{geometry}
\usepackage{hyperref}
\usepackage{float}
\usepackage{caption}
\usepackage{subcaption}

\geometry{margin=1in}

\title{Hessian2DFilters:\\ ...
	A Clean Multiscale Hessian-Based Filtering Framework (2D)}
\author{ }
\date{\today}

\begin{document}
	\maketitle
	
	\section{Overview}
	
	This document describes \texttt{hessian2DFilters}, a clean and well-tested
	implementation of multiscale Hessian-based filters for 2D images.
	
	The framework is designed for \emph{geometric feature detection}, not
	classification or segmentation, and explicitly excludes tensor-based or
	nonlinear multiscale evolution methods such as MFAT or RLINE.
	
	\medskip
	Core principles:
	\begin{itemize}
		\item Stateless per-scale evaluation
		\item Max-over-scales aggregation
		\item Scalar responses derived from Hessian eigenvalues
		\item Clear separation of algorithm families
	\end{itemize}
	
	\section{Supported Filters}
	
	\begin{table}[H]
		\centering
		\caption{Supported Hessian-Based Filters}
		\begin{tabular}{lll}
			\toprule
			\textbf{FilterType} & \textbf{Detects} & \textbf{Typical Applications} \\
			\midrule
			vesselness  & Tubular structures & Blood vessels, pipes \\
			ridge       & Line-like ridges & Curvilinear features \\
			neuriteness & Curvilinear continuity & Neurites, filaments \\
			blob        & Isotropic blobs & Vesicles, spots \\
			plate       & Thick elongated regions & Membranes, bands \\
			\bottomrule
		\end{tabular}
	\end{table}
	
	\noindent
	\textbf{Non-goals:} MFAT, RLINE, tensor FA, scale summation, logical gating.
	
	\section{Algorithm Description}
	
	For each Gaussian scale $\sigma$, the algorithm performs:
	
	\begin{enumerate}
		\item Gaussian smoothing of the image
		\item Computation of the Hessian matrix
		\[
		H =
		\begin{bmatrix}
			I_{xx} & I_{xy} \\
			I_{xy} & I_{yy}
		\end{bmatrix}
		\]
		\item Eigenvalue and eigenvector decomposition
		\item Evaluation of a scalar response function
		\item Max-over-scales aggregation
	\end{enumerate}
	
	This design ensures predictable behavior and avoids inter-scale feedback.
	
	\section{Eigenvalue Convention}
	
	Eigenvalues are interpreted as:
	\[
	|\lambda_1| \le |\lambda_2|
	\]
	
	For bright structures on a dark background, line-like geometry corresponds to:
	\[
	\lambda_2 < 0
	\]
	
	Eigenvalue ordering required by specific filters (e.g.\ Frangi vesselness)
	is enforced locally within the response function.
	
	\section{Filter-Specific Invariants}
	
	\begin{table}[H]
		\centering
		\caption{Mathematical Invariants of Each Filter}
		\begin{tabular}{p{3cm} p{5cm} p{5cm}}
			\toprule
			\textbf{Filter} & \textbf{Guarantees} & \textbf{Does Not Guarantee} \\
			\midrule
			Vesselness & Tubes $>$ background, blob suppression & Thin $>$ thick, binary output \\
			Ridge & Structured regions $>$ background & Blob suppression \\
			Neuriteness & Orientation sensitivity & Blob suppression \\
			Blobness & Blob centers $>$ background & Zero response on lines \\
			Plateness & Thick $>$ thin elongated regions & Blob suppression \\
			\bottomrule
		\end{tabular}
	\end{table}
	
	\section{Parameter Tuning}
	
	\subsection{Scale Selection}
	
	Rule of thumb:
	\[
	\text{structure width} \approx 2\sigma
	\]
	
	\begin{table}[H]
		\centering
		\caption{Typical Scale Ranges}
		\begin{tabular}{ll}
			\toprule
			\textbf{Structure} & \textbf{Suggested $\sigma$ Range} \\
			\midrule
			Thin filaments & $0.5$ -- $2$ \\
			Vessels & $1$ -- $6$ \\
			Thick bands & $3$ -- $8$ \\
			\bottomrule
		\end{tabular}
	\end{table}
	
	\subsection{Filter Parameters}
	
	\begin{itemize}
		\item \textbf{Vesselness:}
		\begin{itemize}
			\item $\beta$ (tube vs blob discrimination): $0.3$--$0.7$
			\item $c$ (noise suppression): $10$--$20$
		\end{itemize}
		\item \textbf{Ridge / Plate:}
		\begin{itemize}
			\item $\alpha$ (anisotropy penalty): $\approx 0.5$
		\end{itemize}
	\end{itemize}
	
	\section{Example Figures}
	
	\subsection{Synthetic Test Image}
	
	\begin{figure}[H]
		\centering
		\includegraphics[width=0.6\linewidth]{figures/test_image.png}
		\caption{Synthetic test image with thin, medium, and thick lines, diagonal
			structures, blobs of varying size, and noise.}
	\end{figure}
	
	\subsection{Filter Responses}
	
	\begin{figure}[H]
		\centering
		\begin{subfigure}{0.45\linewidth}
			\includegraphics[width=\linewidth]{figures/vesselness.png}
			\caption{Vesselness}
		\end{subfigure}
		\hfill
		\begin{subfigure}{0.45\linewidth}
			\includegraphics[width=\linewidth]{figures/ridge.png}
			\caption{Ridge}
		\end{subfigure}
		
		\medskip
		
		\begin{subfigure}{0.45\linewidth}
			\includegraphics[width=\linewidth]{figures/blob.png}
			\caption{Blobness}
		\end{subfigure}
		\hfill
		\begin{subfigure}{0.45\linewidth}
			\includegraphics[width=\linewidth]{figures/plate.png}
			\caption{Plateness}
		\end{subfigure}
		
		\caption{Example responses of Hessian-based filters applied to the test image.
			Each filter highlights different geometric structures.}
	\end{figure}
	
	\section{Unit Testing}
	
	The implementation is accompanied by a comprehensive unit test suite that
	verifies:
	
	\begin{itemize}
		\item Eigenvalue ordering correctness
		\item Filter-specific geometric invariants
		\item Parameter robustness
		\item Performance regression bounds
	\end{itemize}
	
	Tests are executed using:
	\begin{verbatim}
		results = runtests('tests');
	\end{verbatim}
	
	\section{Non-Goals}
	
	This framework does \textbf{not} implement:
	
	\begin{itemize}
		\item MFAT (Alhasson / Obara)
		\item RLINE / flux-based methods
		\item Tensor FA evolution
		\item Logical classification or segmentation
		\item Scale summation or nonlinear inter-scale feedback
	\end{itemize}
	
	Such methods require a fundamentally different algorithmic structure.
	
	\section{References}
	
	\begin{enumerate}
		\item A. F. Frangi et al., \emph{Multiscale Vessel Enhancement Filtering},
		MICCAI, 1998.
		\item Y. Sato et al., \emph{Three-dimensional multi-scale line filter},
		Medical Image Analysis, 1998.
		\item E. Meijering et al., \emph{Design and validation of a tool for neurite tracing},
		Cytometry A, 2004.
		\item T. Lindeberg, \emph{Scale-Space Theory in Computer Vision},
		IJCV, 1998.
	\end{enumerate}
	
\end{document}
